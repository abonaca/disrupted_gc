\documentclass[twocolumn]{aastex63}

% typography
\usepackage[T1]{fontenc}

\setlength{\parindent}{1.\baselineskip}
\newcommand{\acronym}[1]{{\small{#1}}}
\newcommand{\package}[1]{\textsl{#1}}
\newcommand{\gaia}{\textsl{Gaia}}
% \newcommand{\hst}{\textsl{HST}}
% \newcommand{\pans}{\textsl{Pan-STARRS}}

% \newcommand{\deg}{\ensuremath{\textrm{deg}}}
\newcommand{\kpc}{\ensuremath{\textrm{kpc}}}
\newcommand{\kms}{\ensuremath{\textrm{km}\,\textrm{s}^{-1}}}
\newcommand{\masyr}{\ensuremath{\textrm{mas}\,\textrm{yr}^{-1}}}
\newcommand{\feh}{\ensuremath{\textrm{[Fe/H]}}}
\newcommand{\afe}{\ensuremath{\textrm{[$\alpha$/Fe]}}}
\newcommand{\changes}[1]{{\textbf{#1}}}

% aastex parameters
% \received{not yet; THIS IS A DRAFT}
%\revised{not yet}
%\accepted{not yet}
% % Adds "Submitted to " the argument.
% \submitjournal{ApJ}
\shorttitle{}
\shortauthors{bonaca \& kruijssen}

%@arxiver{}
\usepackage{amsmath}

\begin{document}\sloppy\sloppypar\raggedbottom\frenchspacing % trust me

\title{}

\correspondingauthor{Ana~Bonaca}
\email{ana.bonaca@cfa.harvard.edu}

\author[0000-0002-7846-9787]{Ana~Bonaca}
\affil{Center for Astrophysics | Harvard \& Smithsonian, 60 Garden Street, Cambridge, MA 02138, USA}

\author[0000-0002-8804-0212]{J.~M.~Diederik~Kruijssen}
\affiliation{Astronomisches Rechen-Institut, Zentrum f\" ur Astronomie der Universit\" at Heidelberg, M\" onchhofstra\ss e 12-14, D-69120 Heidelberg, Germany}
\affil{Center for Astrophysics | Harvard \& Smithsonian, 60 Garden Street, Cambridge, MA 02138, USA}


\begin{abstract}\noindent % trust me
Numerous stellar streams that have been discovered in the Milky Way as evaporated globular clusters show signs of dynamical perturbation.
N-body models that can illuminate the origin of these perturbations require the cluster's initial mass as a fundamental input parameter.
Here we present orbits and masses of 20 dissolved globular clusters in the Milky Way.
We constrained the streams' orbits by fitting the 3D positions of a stream's endpoints from ground-based photometry and its proper motions from Gaia.
Assuming a dissolution time of 10\,Gyr, we use orbital apocenters and eccentricities to estimate the clusters' initial mass.
Disrupted globular clusters have preferentially lower masses than the surviving population, with the median mass being an order of magnitude smaller.
The overall distribution of apocenters and eccentricities is similar for the disrupted and surviving clusters, however, at a fixed mass disrupted clusters have smaller apocenters and larger eccentricities.
The progenitors of tidal streams observed at the present are a specific, low-mass subset of the initial globular cluster population.
This has implications for establishing the role of internal dynamics in sculpting the observed tidal debris, and the amount of external perturbation, e.g., from dark-matter subhalos, that these streams experienced.
% modeling how much the tidal streams were shaped by nature versus nurture.
% 
% - globular clusters individually: birth sites of binary black holes, some of the oldest stars in the universe
% - collectively: tracers of dark matter halo mass, accretion history
% - mass basic property, but have biased view, bc some tidally disrupted
% - implications for: gc formation mechanisms? 
\end{abstract}

\section{Introduction}
\label{sec:intro}


\section{Stream Orbits}
\label{sec:orbits}
Tidal debris from evaporating globular clusters nearly delineates the progenitor's orbit.
Here we measure orbital parameters of globular clusters dissolved in the Milky Way by fitting orbits to sky positions and distances of 51 thin stellar streams.

- for all streams we have sky positions (stream track) and an average distance
- sky positions are known extremely precisely, and for orbit fitting we assume uncertain to the width, typically $\lesssim0.5^\circ$
- distances are more uncertain because mainly from matched filter, typically $\approx20\,\%$
- sample the stream at $0.5^\circ$, these shown as crosses in figure, colored by distance

- assumed gravitational potential
- orbit represented as a 6D point, initialized xdeg from the end of the stream, at the average distance, with the velocity vector pointing to the opposite end of the stream
- final free parameter: orbital length
- priors: flat (physical -- distance positive, velocity lower than 600km/s?), stream length to match data to xdeg
- preliminary solution w minimize
- start ball there, sample w emcee, 64 walkers advanced for x steps, discard first y steps as burn-in
- convergence test: just no motion in median, dispersion?

- results in figure: lines samples from the posterior, also colored by distance
- zoom in: track fit very well, orbit fits expect some distance gradients, can be improved in the future w better selected members
- orbit xGyr in the future, on average y orbital periods, that we use to measure apocenter $R_{apo}$, eccentricity, $e=(R_{apo} - R_{peri})/(R_{apo} + R_{peri})$
- within one orbit peri/apo precisely measured (fractional precision x), across samples, median/90 percentile fractional uncertainty for apocenter, eccentricity
- two examples: different kinds of orbits, same precision?


\section{Masses of Disrupted Globular Clusters}
\label{sec:disrupted}


\section{Discussion}
\label{sec:discussion}
- summary

- caveats:
-- orbit fits -> compare to published ones; 5d fits predictive, don't change much w 6d (ibata)
-- 10 Gyr disruption time for estimating the mass

% - check how sensitive the inferred rapo is on the details of the stream track used
% - for streams w published

- compare to mass estimates in the detect part of the stream -> are they mostly detected or are they extending past the currently detected endpoints?

- lower mass -> implications for epicycles

\vspace{0.5cm}
It is a pleasure to thank:

\software{
\package{Astropy} \citep{astropy, astropy:2018},
\package{gala} \citep{gala},
\package{IPython} \citep{ipython},
\package{matplotlib} \citep{mpl},
\package{numpy} \citep{numpy},
\package{scipy} \citep{scipy}
}

\bibliographystyle{aasjournal}
\bibliography{disrupted_gc}


\end{document}